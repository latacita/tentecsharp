\documentclass{sig-alt-full}

\usepackage{graphicx}
\usepackage{url}

\begin{document}

\newcommand{\imp}[1]{{\small{\sf #1}}}
\newcommand{\stereotype}[1]{$<<${\small{\sf #1}}$>>$}

% --- Author Metadata here ---
\conferenceinfo{FOSD'10,}{October 13, 2010, Eindhoven (The Netherlands)}
\CopyrightYear{2010} \crdata{...}
% --- End of Author Metadata ---

\title{C\# Partial Classes as a Mechanisms to Implement Feature-Oriented Designs: An Exploratory Study\thanks{This work has been supported by the Spanish Ministry Project TIN2008-01942/TIN, the EC STREP Project AMPLE IST-033710, and the Junta de Andaluc{\'i}a regional project FarmWare TIC-5231.}}

\numberofauthors{3}

\author{
% 1st. author
\alignauthor Lidia Fuentes \\
       \affaddr{Dpto. Lenguajes y Ciencias de la Computaci{\'o}n}\\
       \affaddr{Universidad de M{\'a}laga (Spain)}\\
       \email{lff@lcc.uma.es}
% 2nd. author
\alignauthor Elio L�pez \\
       \affaddr{Dpto. Lenguajes y Ciencias de la Computaci{\'o}n}\\
       \affaddr{Universidad de M{\'a}laga (Spain)}\\
       \email{ealopezs@gmail.com}
% 3rd. author
\alignauthor Pablo S{\'a}nchez \\
       \affaddr{Dpto. Matem{\'a}ticas, Estad{\'i}stica y Computaci{\'o}n}\\
       \affaddr{Universidad de Cantabria (Spain)}\\
       \email{p.sanchez@unican.es}
}

\maketitle

\begin{abstract}
    %=========================================================================%
% Author: Pablo S�nchez                                                   %
% Paper: FOSD2010 (abstract)                                              %
% Version: 1.0                                                            %
% Date   : 2010/05/07                                                     %
%=========================================================================%

C\# partial classes allows developers to divide the implementation of a class into several slices where each slice contains an increment of functionality as compared to the other slices. Thus, combining different set of slices, we can get classes with a variable range of functionality. With this description, C\# partial classes seems to be, as also pointed out by other authors, a suitable mechanism for implementing feature-oriented designs. This paper explores this idea, by systematically applying C\# to a feature-oriented decomposition based on an industrial case study and comparing the results we have previously obtained using the feature-oriented language CaesarJ. As main contributions, (1) we identify benefits and pitfalls of C\# partial classes for implementing feature-oriented decompositions; and (2) we outline potential solutions to alleviate these pitfalls. 
\end{abstract}

\category{D2.3}{Software-Engineering}{Coding Tools and Techniques}

\terms{Experimentation, Languages}

\keywords{Partial Classes, Feature-Oriented Programming, Software-Product Line, C\#}

%\begin{figure*}[!tb]
%  \begin{center}
%    \includegraphics[width=.80\linewidth]{images/robotExample.eps}
%    \caption{Robot example}
%    \label{fig:example}
%  \end{center}
%\end{figure*}


\section{Introduction}
\label{sec:introduction}

%%==================================================================%%
%% Author : Abascal Fern�ndez, Patricia                             %%
%%          S�nchez Barreiro, Pablo                                 %%
%% Version: 2.1, 14/06/2013                                         %%                                                                                    %%                                                                  %%
%% Memoria del Proyecto Fin de Carrera                              %%
%% Archivo ra�z                                                     %%
%%==================================================================%%

\chapterheader{Introducci�n}{Introducci�n}
\label{chap:introduction}

Este cap�tulo sirve de introducci�n a la presente Memoria de Proyecto Fin de Carrera. Para ello, en primer lugar se describe el contexto general donde se enmarca dicho proyecto y que da lugar al mismo. Se describe luego, a grandes rasgos, el proyecto para la metodolog�a Te.Net, proyecto general de amplio alcance donde se inscribe el presente proyecto. A continuaci�n, se exponen los objetivos principales del proyecto. Por �ltimo, se describe c�mo se estructura el presente documento.

\chaptertoc

\section{Contexto del Proyecto}
\label{sec:intr:introduction}

%%==================================================================%%
%% Author : Abascal Fern�ndez, Patricia                             %%
%%          S�nchez Barreiro, Pablo                                 %%
%% Version: 1.2, 23/04/2013                                         %%                                                                                    %%                                                                  %%
%% Memoria del Proyecto Fin de Carrera                              %%
%% Introducci�n                                                     %%
%%==================================================================%%

El principal objetivo de este Proyecto de Fin de Carrera es implementar un conjunto de generadores de c�digo que permitan transformar modelos UML orientados a caracter�sticas en c�digo C\#. Para dar soporte a la orientaci�n a caracter�sticas a nivel de c�digo C\#, se utilizar� el patr�n de dise�o
\emph{Slicer}. Dicho patr�n fue espec�ficamente para tal prop�sito como parte de otro Proyecto Fin de Carrera presentado en esta misma Facultad~\cite{}. 

La \emph{orientaci�n a caracter�sticas}~\cite{} tiene como objetivo  encapsular porciones coherentes de la funcionalidad proporcionadas por una aplicaci�n en m�dulos independientes llamados \emph{caracter�sticas}. La orientaci�n a caracter�sticas eleva el nivel al cual se agrupa la funcionalidad de un sistema del concepto de clase al concepto de \emph{conjunto} o \emph{familia de clases}, las cuales se a�aden o eliminan de una aplicaci�n como un todo. 

De esta forma, podemos obtener productos con funcionalidades ligeramente diferentes mediante la simple incorporaci�n o eliminaci�n de m�dulos representando caracter�sticas. 

Lo que convierte la obtenci�n de diferentes versiones de una misma aplicaci�n combinando diferentes conjuntos de caracter�sticas en una tarea sencilla. Los \emph{dise�os orientados a caracter�sticas} deben asegurar que el resultado de la composici�n de un conjunto de caracter�sticas produce como resultado una aplicaci�n correcta y segura.


La orientaci�n a caracter�sticas~\cite{} se utiliza frecuentemente como mecanismo de dise�o e implementaci�n de las conocidas como \emph{L�neas de Productos Sw}~\cite{}.

El objetivo de una \emph{L�neas de Productos Sw}~\cite{} es ... \todo{Copiar la definici�n del proyecto de Alejandro o de Daniel}.

%%===================================================================%%
%% NOTA(Pablo): Establecer relaci�n entre ambos paradigmas           %%
%%===================================================================%%

Este Proyecto Fin de Carrera se enmarca dentro un proyecto general y m�s ambicioso

Dichos generadores de c�digo se integrar�an en una metodolog�a m�s amplia para el desarrollo de \emph{L�neas de Productos Software}~\cite{}, denominada Te.Net.


es integrar dichos generadores en la metodolog�a de desarrollo de \emph{L�neas de Productos Software}~\cite{} denominada TE.NET, una versi�n para la plataforma .NET de la metodolog�a TENTE~\cite{}. A continuaci�n intentaremos introducir de forma breve al lector en estos conceptos.



Dada la cantidad de terminolog�a novedosa contenida en la descripci�n del proyecto, procedemos a describir brevemente la historia precedente a la gestaci�n del mismo.


%%===================================================================%%
%% NOTA(Pablo): Esto se mueve mejor a antecedentes                   %%
%%===================================================================%%
%%
%% El uso de las l�neas de producto software permite la reducci�n
%% de costes de desarrollo por la reutilizaci�n de la tecnolog�a en
%% los distintos sistemas, a mayor cantidad de productos a desarrollar
%% mayor rentabilidad respecto a los sistemas creados individualmente.
%% Ofrece alta calidad en el producto resultante porque se realizan
%% pruebas de los componentes de la plataforma en diferentes tipos
%% de producto para ayudar a detectar y corregir errores. Reduce el
%% tiempo de creaci�n debido a la reutilizaci�n de los componentes ya
%% existentes para cada nuevo producto y reduce tambi�n el esfuerzo
%% requerido por el mantenimiento ya que cuando se cambia algo de un
%% componente de la plataforma, ese cambio se propaga a todos los
%% componentes que lo empleen, y de esta forma se reduce el esfuerzo
%% de aprender c�mo funciona cada elemento individualmente.
%%
%% En contraposici�n a la flexibilidad que ofrece el desarrollo de
%% software individual, espec�fico para cliente pero que supone grandes
%%  costes, las l�neas de producto software delimitan las variaciones
%% de sus productos a un conjunto prefijado y optimizan, por tanto, los
%% procesos para dichas variaciones.
%%
%%
%% La l�nea de productos software se puede extrapolar a otros �mbitos de
%% producci�n. Un ejemplo cl�sico de l�nea de productos es la fabricaci�n
%% de autom�viles, donde se ofrece al cliente un modelo base al cual puede
%% a�adir aquellos extras que as� desee, personalizando el veh�culo y
%%  adapt�ndolo a sus necesidades. De esta forma partiendo del mismo modelo
%% y de unas variaciones adicionales preestablecidas, y dise�adas de tal
%% forma que se adaptan perfectamente al modelo seleccionado, se puede
%% obtener gran cantidad de variaciones en el modelo final de manera
%% autom�tica.

En el �mbito del desarrollo software, las empresas ya no se centran en la creaci�n de un producto espec�fico para un cliente (por ejemplo, dise�ar y construir un portal para la Universidad de Cantabria), sino en un domino (por ejemplo, dise�ar y construir un portal para universidades). Los principales desaf�os a los que se enfrentan las empresas son: delimitar dicho dominio, identificar las distintas variaciones que se van a permitir y desarrollar la infraestructura que permita realizar los productos a bajo coste sin reducir la calidad.


%%%%%%%%%%%% Metodolog�as de desarrollo de l�neas de productos software %%%%%%%%%%%%

El proceso de desarrollo de la l�nea de productos software se divide en dos procesos \cite{pohl:2010}: Ingenier�a de Dominio e Ingenier�a de la Aplicaci�n. Por un lado la \emph{Ingenier�a del Dominio} se encarga de la construcci�n de la plataforma mediante la delimitaci�n del conjunto de aplicaciones para las que est� creada, adem�s de definir y construir qu� caracter�sticas ser�n reusables y cuales espec�ficas para cada uno de los productos que se desean fabricar.

Por otra parte, la \emph{Ingenier�a de la Aplicaci�n} se encarga de la creaci�n de los productos para clientes concretos. Partiendo de la plataforma creada en la fase de Ingenier�a de Dominio, y reutilizando tantos componentes como fuera necesario, se crea una especializaci�n del producto base acorde a los requisitos del cliente.


%%%%%%%%%%%% Clases parciales y patr�n Slicer %%%%%%%%%%%%
Tal como se ha descrito al inicio de este apartado, el objetivo del presente Proyecto Fin de Carrera consiste en el desarrollo e implementaci�n de unos generadores de c�digo que permitan la tranformaci�n del dise�o de los modelos en una implementaci�n en c�digo C\# de dichos dise�os, para ello se usar�n las prestaciones que ofrecen el uso de las clases parciales del lenguaje C\# basadas en el patr�n Slicer.

Las \emph{clases parciales} permiten a los desarrolladores fragmentar la implementaci�n de una clase en un conjunto de ficheros, cada uno de los cuales contiene una porci�n, o incremento, de una funcionalidad de la clase. Sin embargo, no ofrecen ning�n mecanismo para agrupar o encapsular caracter�sticas, por lo que no es posible ocultar clases y m�todos que pertenecen a una caracter�stica espec�fica de aquellas clases y m�todos que pertenecen a caracter�sticas independientes. Adem�s, permiten a�adir nuevos atributos y m�todos a existentes clases parciales pero no permite sobreescribir o extender m�todos ya existentes.

Para solventar dichos problemas, el profesor Pablo S�nchez, dentro del Departamento de Matem�ticas, Estad�stica y Computaci�n, ha desarrollado un patr�n de dise�o llamado \emph{Patr�n Slicer} \cite{perez:2011} que parte de la siguiente idea: todos los problemas que se pretenden solucionar tienen origen en el hecho de no poner tener m�todos con el mismo nombre en distintas clases parciales, hay que evitar dicha situaci�n. Estos fragmentos de clases parciales, son combinados en tiempo de compilaci�n para crear una �nica clase que auna todas las caracter�sticas seleccionadas inicialmente por el cliente.

Por ejemplo, supongamos que un cliente quiere un veh�culo con varias caracter�sticas adicionales entre las que se encuentran: aire acondicionado, sensor de lluvia, medidor de temperatura en grados Celsius y GPS integrado en idioma espa�ol e ingl�s. La base de nuestro producto final ser� el veh�culo, al cual iremos a�adiendo las distintas caracter�sticas requeridas por el cliente. Hay algunas peculiaridades, la clase del medidor de temperatura puede estar a su vez fragmentada en varios componentes (temperatura en Celsius, temperatura en Farentheit) y de los cuales en el modelo final solo usaremos uno de ellos, el de temperatura en Celsius. Lo mismo ocurre con el selector de idiomas para el GPS, solo se elegir� el idioma espa�ol e ingl�s. De esta forma, el producto final juntar� todas estas caracter�sticas dentro de un mismo elemento que ser� el veh�culo entregado al usuario final atendiendo a sus requisitos.

%%%%%%%%%%%% Retoma el objetivo del proyecto %%%%%%%%%%%%
El objetivo de este Proyecto Fin de Carrera es implementar generadores de c�digo que abordar�n tanto la implementaci�n de la familia de productos software cubierta por la l�nea de productos, como la configuraci�n de productos concretos pertenecientes a dicha familia utilizando las prestaciones de las clases parciales en C\# y el Patr�n Slicer. Con esto esperamos haber aclarado el primer p�rrafo de esta secci�n al lector no familiarizado con las l�neas de productos software, clases parciales en lenguaje C\# y/o el Patr�n Slicer.

%%%%%%%%%%%%%%%%%%%%% Sin modificar del fichero original
Tras esta introducci�n, el resto del presente cap�tulo se estructura como sigue: La Secci�n []  proporciona []. Por �ltimo, la Secci�n~\ref{sec:intr:estructura} describe la estructura general del presente documento.


\section{La metodolog�a Te.NET}

%%==================================================================%%
%% Author : Abascal Fernández, Patricia                             %%
%%          Sánchez Barreiro, Pablo                                 %%
%% Version: 1.3, 18/06/2013                                         %%                                                                                    %%                                                                  %%
%% Memoria del Proyecto Fin de Carrera                              %%
%% Introduccion/Metodologia TeNet                                   %%
%%==================================================================%%

Tal como se ha comentado en la sección anterior, la metodología Te.Net se trata de una variante de la tecnología TENTE. A diferencia de TENTE, la cual obliga a utilizar como lenguaje de programación final un lenguaje orientado a características que soporte el concepto de \emph{familia de clases}, al estilo de \emph{CaesarJ}~\citep{ivica:2006} u \emph{ObjectTeams}~\citep{stephan:2002}, Te.NEt utiliza como lenguaje de programación destino un lenguaje convencional orientado a objetos, más concretamente C\#.

El primer paso a realizar para llevar a cabo este rediseño de la metodología TENTE era analizar cómo se podía dar soporte a la orientación a aspectos en un lenguaje de programación orientado a objetos como C\#. Tras realizar una buscar opciones en el estado del arte actual, se encontró un prometedor trabajo~\citep{perez:2011} en el cual se proponía la utilización de las clases parciales de C\# como mecanismos para dar soporte a la orientación características.

%%==================================================================%%
%% NOTA(Pablo): Esto se pasaría a la parte de antecedentes           %%
%%==================================================================%%
%%
%% Las \emph{clases parciales} permiten a los desarrolladores fragmentar %% la implementación de una clase en un conjunto de ficheros, cada uno
%% de los cuales contiene una porción, o incremento, de una
%% funcionalidad de la clase. Sin embargo, no ofrecen ningún mecanismo
%% para agrupar o encapsular características, por lo que no es posible
%% ocultar clases y métodos que pertenecen a una característica
%% específica de aquellas clases y métodos que pertenecen a
%% características independientes. Además, permiten añadir nuevos
%% atributos y métodos a existentes clases parciales pero no permite
%% sobreescribir o extender métodos ya existentes.
%%
%%==================================================================%%

Por tanto, se decidió evaluar dicho trabajo en profundidad con objeto de verificar las ideas propuestas en el mismo. Los experimentos realizados~\citep{sanchez:2010} revelaron diferentes debilidades de las clases parciales como mecanismo para la implementación de líneas de productos software.

Para solventar los problemas detectados, se creó, como resultado de otro Proyecto Fin de Carrera presentado en esta misma Facultad, un patrón de diseño denominado \emph{Slicer Pattern}~\citep{perez:2011}. Dentro de dicho Proyecto Fin de Carrera se implementó una línea de productos software para el desarrollo de software de gestión de hogares inteligentes.

Una vez que se había solventado el problema de cómo soportar la orientación a características en C\#, la siguiente tarea a realizar era la de adaptar los generadores de códigos originales para que soportasen la generación de código en C\# en lugar de CaesarJ. Esta tarea constituye el objetivo principal de este proyecto, el cual se detalla en la siguiente sección.




\section{Motivaci�n y Objetivos}
\label{sec:intr:planning}

%%==================================================================%%
%% Author : Tejedo Gonz�lez, Daniel                                 %%
%%          S�nchez Barreiro, Pablo                                 %%
%% Version: 1.0, 14/11/2012                                         %%                   %%                                                                  %%
%% Memoria del Proyecto Fin de Carrera                              %%
%% Introducci�n/Introducci�n                                        %%
%%==================================================================%%

Como ya se ha comentado en la secci�n de introducci�n, no existe ninguna herramienta que posea de forma conjunta una serie de elementos de inter�s para el modelado de L�neas de Productos Software y �rboles de Caracter�sticas. M�s concretamente, no existe ninguna herramienta que contemple el modelado, configuraci�n y validaci�n de caracter�sticas clonables. Estas caracter�sticas son imprescindibles para el modelado de la variabilidad estructural. Por lo tanto, el objetivo de Hydra siempre fue suplir esas carencias, en la medida de lo posible.

Concretando m�s en concreto, los objetivos de Hydra se pueden clasificar en los 4 que se enumeran a continuaci�n: \\

1. Desarrollar un editor completamente gr�fico y amigable al usuario para la construcci�n de modelos de caracter�sticas, incluyendo soporte para el modelado de caracter�sticas clonables.

2. Desarrollar un editor textual y una sintaxis propia para la especificaci�n de restricciones entre caracter�sticas, incluyendo restricciones que involucren caracter�sticas clonables.

3. Desarrollar Un editor gr�fico, asistido y amigable al usuario para la creaci�n de configuraciones de modelos de caracter�sticas, incluyendo soporte para la configuraci�n de caracter�sticas clonables.

4. Crear un validador que compruebe que las configuraciones creadas satisfacen las restricciones definidas para el modelo de caracter�sticas, incluso cuando estas restricciones contengan caracter�sticas clonables. \\

La labor a desarrollar dentro del marco concreto de este proyecto de fin carrera fue continuar el proyecto Hydra donde se hab�a dejado anteriormente, es decir, una vez los objetivos 1 y 3 hab�an sido cumplimentados, pasar a implementar la funcionalidad correspondiente a los objetivos 2 y 4. Para satisfacer dichos objetivos, se realizaron las tareas que se describen a continuaci�n: \\

1. Estudio del estado del arte. El objetivo de esta fase es adquirir los conceptos necesarios para comprender el contexto del proyecto Hydra, as� como los necesarios para continuar desarrollando la aplicaci�n en el punto en que fue visitada por �ltima vez. M�s concretamente, ha sido fundamental familiarizarse con los conceptos de L�nea de Producto Software, �rbol de Caracter�sticas (con y sin caracter�sticas clonables) y de Ingenier�a Dirigida por Modelos en general, y de Ingenier�a de Lenguajes Dirigida por Modelos en particular.

2. Estudio de las herramientas utilizadas. El objetivo de esta fase comprende la familiarizaci�n con todas las herramientas y tecnolog�as necesarias para desarrollar la parte estipulada de la aplicaci�n. En concreto, con EMF, Ecore, EMFText, Eclipse Validation Framework, Eclipse Plugin Development y Subversion.

3. Desarrollo de un editor de restricciones externas entre caracter�sticas. El objetivo de este editor es soportar la especificaci�n de restricciones externas ante un modelo de caracter�sticas proporcionado por el usuario. Tales restricciones son expresiones similares a f�rmulas l�gicas, salvo por alguna peculiaridad espec�fica. Es por eso que se opt� por el uso de un editor textual en lugar de uno gr�fico, ya que es el m�todo m�s habitual de representar este tipo de operaciones. Para crear el metamodelo del lenguaje se ha utilizado la herramienta Ecore, mientras que para definir la gram�tica se ha utilizado EMFText. 

4. Desarrollo de un validador de configuraciones. Una vez se finaliz� de crear el editor para las restricciones, el siguiente paso l�gico era aportarle una sem�ntica que permitiera comprobar si las restricciones creadas satisfacen la configuraci�n proporcionada por el usuario. Para implementar la sem�ntica se utilizaron las herramientas EMF, Eclipse Validation Framework y Eclipse Plugin Development. 

5. Validaci�n y pruebas. Con objeto de evaluar, probar y verificar el correcto funcionamiento de nuestra herramienta se han sometido algunas configuraciones del �rbol de caracter�sitcas Smarthome a una serie de pruebas de caja negra, tratando de probar todas las operaciones de restricciones posibles en todos los contextos problem�ticos y habituales.  


\section{Estructura del Documento}
\label{sec:intr:estructura}

%%==================================================================%%
%% Author : Abascal Fern�ndez, Patricia                             %%
%%          S�nchez Barreiro, Pablo                                 %%
%% Version: 1.3, 18/06/2013                                         %%                                                                                    %%                                                                  %%
%% Memoria del Proyecto Fin de Carrera                              %%
%% Introducci�n/Roadmap                                             %%
%%==================================================================%%

Tras este cap�tulo introductorio, el resto del documento se estructura como sigue. El Cap�tulo~\ref{chap:background} describe brevemente los conceptos necesarios para poder entender la presente memoria, y que no se pueden presuponer conocidos por el lector, tales como qu� es una \emph{L�nea de Producto Software} o en qu� consiste el \emph{Slicer Pattern}. El Cap�tulo~\ref{chap:domain} explica el proceso de desarrollo de uno de los generadores de c�digo creados, concretamente el que act�a durante la fase de la fase de \emph{Ingenier�a del Dominio} del desarrollo de una l�nea de productos software. El Cap�tulo~\ref{chap:application} describe el desarrollo del generador de c�digo que act�an durante la fase de configuraci�n de una l�nea de productos software, la \emph{Ingenier�a de Aplicaciones}. Dicho cap�tulo tambi�n comenta brevemente las acciones realizadas para el despliegue de la aplicaci�n. Por �ltimo, el Cap�tulo~\ref{chap:conclusiones} sirve de sumario y cierre a esta memoria de Proyecto Fin de Carrera, proporcionando tambi�n las conclusiones extra�das tras su realizaci�n, as� como posibles trabajos futuros.





\section{Case Study: A Smart Home Software Product Line}
\label{sec:caseStudy}

Copia y pega de la descripci�n de la Smart Home m�s breve explicaci�n de los modelos arquitect�nicos. La figura del modelo hay que ponerla despu�s de escribir el art�culo, para que contenga todos los elementos que nos hacen falta para explicar los fallos de C\#.


\section{What we want to find in a feature-oriented programming language}
\label{sec:fopFeatures}

B�sicamente, caracter�sticas que tienen los lenguajes como CaesarJ que nos facilitan la vida para implementar modelos de SPL, tales como lo de reescritura autom�tica de las dependencias entre clases.


\section{Implementing the Smart Home models using C\# partial classes}
\label{sec:partialClasses}

Describir como se implementan diferentes situaciones en el modelo de la SmartHome.

\subsection{Scenarios 1: ...}

Descripci�n de como implementar algo tal como una feature que extiende de otra.

\subsection{Scenarios 2: ...}

Descripci�n de como implementar otra cosa, tal como una feature que extiende de dos.

\subsection{Scenarios 3: ...}

Otro caso diferente ...

\subsection{Conclusions}

Tabla o bullets comparando lo que querr�amos tener y lo que tenemos con las clases parciales.

\section{Sketch of solutions for C\# partial classes pitfalls}

Breves comentarios sobre como solucionar estos problemas. (NOTA: No estoy muy seguro acerca de si poner o no esta secci�n).

\section{Summary and Future Work}
\label{sec:summary}

\bibliographystyle{plain}
\bibliography{fosd2010}

\balancecolumns

\end{document}

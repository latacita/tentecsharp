%=========================================================================%
% Author : Alejandro P�rez Ruiz                                           %
% Author : Pablo S�nchez Barreiro                                         %
% Version: 1.1, 10/06/2011                                                %
% Master Thesis: Background/PartialClasses                                % 
%=========================================================================%

Las clases parciales C\# \cite{albahari:2010} nos permiten dividir la implementaci�n de una clase, estructura o interfaz en varios archivos de c�digo fuente. Cada fragmento representa una parte de la funcionalidad global de la clase. Todos estos fragmentos se combinan en tiempo de compilaci�n para crear una �nica clase, la cual contiene toda la funcionalidad especificada en las clases parciales. Por lo tanto,las clases parciales C\# pueden utilizarse como un mecanismo adecuado para implementar caracter�sticas, dado que cada incremento en funcionalidad para una clase se podr�a encapsular en una clase parcial separada.


Para poder ser compiladas y agrupadas en una sola clase, todas las clases parciales deben pertenecer al mismo espacio de nombres, poseer la misma visibilidad y deben ser declaradas con el indicador clave partial. En C\#, un espacio de nombre es simplemente empleado para agrupar clases relacionadas y evitar conflictos de nombres. Para especificar los archivos C\# que deben ser incluidos en una compilaci�n, se emplea un documento XML que contiene informaci�n acerca del proyecto y que especifica que ficheros deben ser compilados para generar el proyecto. Por lo tanto, es posible incluir y excluir f�cilmente la funcionalidad encapsulada dentro de una clase parcial simplemente a�adiendo o eliminando dicha clase parcial de este fichero XML.

Para ilustrar lo dicho anteriormente, se ha vuelto a utilizar el problema de las expresiones implement�ndolo con clases parciales. La figura \ref{back:fig:partialClass} muestra como hemos excluido de la compilaci�n la caracter�stica que representa la operaci�n de imprimir una expresi�n en formato infijo.
Este mecanismo de clases parciales permite a�adir o compartir funcionalidad entre un conjunto de clases que no precisan estar relacionadas mediante ning�n tipo de relaci�n jer�rquica, tal como ocurre con la herencia.

\begin{figure}[!tb]
\begin{center}
\begin{footnotesize}
\begin{verbatim}
01    <itemgroup>
02    <!--Eval-->
03    <Compile Include="Eval\Add.cs" />
04    <Compile Include="Eval\IExpressionsEval.cs" />
05    <Compile Include="Eval\IExpressions.cs" />
06    <Compile Include="Eval\Integer.cs" />
07    <Compile Include="Eval\Mult.cs" />
08    <!--Infix-->
09    <!--<Compile Include="Infix\Add.cs" />
10    <Compile Include="Infix\IExpressionInfix.cs" />
11    <Compile Include="Infix\IExpressions.cs" />
12    <Compile Include="Infix\Integer.cs" />
13    <Compile Include="Infix\Mult.cs" />-->
14    ...
15    </itemgroup>
16    </Project>
\end{verbatim}
\end{footnotesize}
\end{center}
\caption{Implementaci�n del archivo XML que guarda la informaci�n para la compilaci�n}
\label{back:fig:partialClass}
\end{figure}

Por lo tanto, algunos autores \cite{laguna:2007} sostienen que, las clases parciales C\# representan una alternativa a la herencia m�ltiple para manejar variabilidad relacionada con programaci�n orientada a caracter�sticas.

% Qu� hace esto aqu�

% Explicar que son brevemente, y ejemplo usando las expresiones

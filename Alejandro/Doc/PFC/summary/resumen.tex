%=============================================================================%
% Author : Alejandro P�rez Ruiz                                               %
% Author : Pablo S�nchez Barreiro                                             %
% Version: 1.0, 07/03/2011                                                    %
% Master Thesis: Resumen                                                      %
%=============================================================================% 

\section{Resumen}

El objetivo de una l�nea de productos software \cite{pohl:2005} es crear una infraestructura adecuada a partir de la cual se puedan derivar, tan autom�ticamente como sea posible, productos concretos pertenecientes a una familia de productos software. Una familia de productos software es un conjunto de aplicaciones software similares, que por tanto comparten una serie de caracter�sticas comunes, pero que tambi�n presentan variaciones entre ellos.

El software para el control de hogares inteligentes o automatizados es un claro ejemplo de dominio donde un enfoque de l�neas de productos software resulta muy adecuado. Dicho software presenta un amplio rango de variaciones debido a los diferentes dispositivos que pueden ser controlados en cada tipo de hogar (p. ej.: ventanas, puertas, luces, radiadores, etc.) y las funciones que se desean que dichos dispositivos cumplan (p. ej.: simulaci�n de presencia, encendido y apagado autom�tico de luces, control inteligente de la energ�a, etc.)

El objetivo del proyecto es una l�nea de productos software para hogares autom�ticos y/o inteligentes, de forma que se pueda automatizar el proceso de desarrollo de aplicaciones concretas para hogares espec�ficos.  Dicha infraestructura se desarrolla en la plataforma .NET, usando Visual Studio 2010. Se usan las clases parciales de C\# como principal mecanismo para la modularizaci�n, composici�n y gesti�n de las diferentes caracter�sticas variables que conformar�n la l�nea de productos software. La descripci�n o especificaci�n informal de requisitos que debe cumplir la l�nea de productos est� basada en un caso de estudio industrial proporcionado por Siemens AG.

De tal modo, que se han desarrollado dos plugins para el entorno de desarrollo Visual Studio 2010, que permiten a los usuarios modelar hogares autom�ticos y/o inteligentes seleccionando las caracter�sticas que mejor se adapten a sus requerimientos. Y a trav�s de los modelos creados se derivan autom�ticamente las aplicaciones concretas para cada modelo.

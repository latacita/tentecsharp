%==================================================================%
% Author : P�rez Ruiz, Alejandro                                   %
% Version: 1.0, 16/03/2011                                         %                                                                                    %                                                                  %
% Memoria del Proyecto Fin de Carrera                              %
% Cap�tulo Conclusiones y Trabajos Futuros                         %
%==================================================================%

\chapterheader{Conclusiones y Trabajos Futuros}{Conclusiones y Trabajos Futuros}

\label{chap:conclusiones}
%Introducci�n al cap�tulo

\chaptertoc
\section{Conclusiones}
%%% Qu� hace exactamente mi proyecto
Este proyecto ha desarrollado una l�nea de productos software para hogares autom�ticos y/o inteligentes mediante la tecnolog�a .NET en el lenguaje C\# a trav�s de sus clases parciales. El principal objetivo que se busca cuando se realiza un tipo de proyecto como este, es el de reducir costes de desarrollo y mejorar la calidad de los productos software. No obstante, el presente proyecto no ten�a como principal cometido desarrollar un l�nea de productos software que pudiese ser trasladada a un caso pr�ctico real de manera inmediato. En cierto modo, se ha utilizado un caso de estudio como es el de la automatizaci�n de los hogares, debido a la gran variabilidad y elementos comunes que poseen, para encontrar las posibles ventajas y desventajas que se presentan cuando se trabaja con las clases parciales de C\# sobre la tecnolog�a .NET para construir l�neas de productos software.

Durante el desarrollo del proyecto se han encontrado problemas que se presentan cuando se utilizan las clases parciales como mecanismo principal para encapsular las distintas caracter�sticas de una l�nea de productos. Por ello se han desarrollado nuevos mecanismos que pueden ser utilizados como una alternativa para suplir estos problemas. Por lo que el principal objetivo ha sido demostrar que utilizando los mecanismos desarrollados en el presente proyecto se pueden construir l�neas de productos software bajo la tecnolog�a .NET y las clases parciales de C\#.

Tambi�n ha quedado demostrado que es posible construir software dirigido por modelos en una l�nea de productos software, combinando las herramientas denominadas \emph{DSL Tools} y las clases parciales de C\# bajo el mismo entorno de desarrollo. Consiguiendo dirigir el desarrollo software a trav�s de un nivel de abstracci�n alto a otro nivel inferior, hasta llegar al c�digo fuente. No obstante, existen ciertos inconvenientes al utilizar las herramientas \emph{DSL Tools} para definir metamodelos, debido a que pueden provocar un sustancial aumento en esfuerzo y tiempo. Es un factor a tener muy en cuenta, especialmente en proyectos de gran magnitud. Esto es debido principalmente, a la informaci�n mal estructurada que normalmente se encuentra cuando se trata de documentarse acerca de dichas herramientas.

Resumiendo, el presente proyecto puede considerarse como un referente para la creaci�n de l�neas de productos software bajo la tecnolog�a .NET. De tal modo, que se han desarrollado m�todos que permiten suplir las carencias que se encuentran al utilizar esta tecnolog�a con las l�neas de productos software. Permitiendo a los numerosos desarrolladores de software y empresas que trabajan bajo esta tecnolog�a, utilizar las l�neas de productos software, sin incurrir en nuevos gastos de aprendizaje, cambio de herramientas...

%%% Qu� no hace

%%% Experiencias Personales

\section{Trabajos Futuros}
%%% Cualquier cosa que se puede hacer para mejorar este proyecto
Todo proyecto es siempre susceptible de mejorar, existiendo una serie de caracter�sticas, que bien por falta de tiempo, por su excesiva complejidad o por ser de un inter�s secundario, nunca fueron incorporadas al proyecto o quedaron por resolver. Y como este proyecto no es ninguna excepci�n, a continuaci�n se muestra una lista de trabajos futuros que se podr�an realizar:

\begin{enumerate}
\item 

\item

\item

\end{enumerate}



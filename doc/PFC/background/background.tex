%=============================================================================%
% Author : Alejandro P�rez Ruiz                                               %
% Author : Pablo S�nchez Barreiro                                             %  % Version: 1.0, 07/03/2011                                                    %
% Master Thesis: Background, master file                                      %
%=============================================================================%

\chapterheader{Antecedentes}{Antecedentes}
\label{chap:introduction}

% Introducci�n al cap�tulo

\chaptertoc

\section{L�neas de Productos Software}

% Explica que es una l�nea de productos software, objetivos y terminolog�a
% \footnote{} 

\section{Programaci�n Orientada a Caracter�sticas}

% Objetivos de la orientaci�n a caracter�sticas

\subsection{Limitaciones de la Orientaci�n a Objetos para la Programaci�n Orientada a Caracter�sticas}

% Limitaciones de los lenguajes OO para FOP
% - Cuentas el ejemplo de las expresiones
% - Muestras la soluci�n en C#.
% - Resumes los problemas

\subsection{Ventajas de los lenguajes orientados a caracter�sticas}

% P�rrafo introducci�n
% Introducci�n CaesarJ 
% Ejemplo en CeasrJ, resaltando ventajas

\section{Clases Parciales C#}

% Qu� hace esto aqu�

% Explicar que son brevemente, y ejemplo usando las expresiones

\section{Caso de Estudio: Hogares \emph{Inteligentes} Automatizados}

% Resumir lo que aparece en diversos documentos


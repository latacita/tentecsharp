%=============================================================================%
% Author : Alejandro P�rez Ruiz                                               %
% Author : Pablo S�nchez Barreiro                                             %  % Version: 2.0, 23/02/2011                                                    %
% Master Thesis: Introduction                                                 %
%=============================================================================%


%%% Schema to write a paper introduction
%% Description of Purpose
	% What problem, issue or question does this research address ?
		%
	% What limitations or failings of current understanding, knowledge, method,
	% or technologies does this research resolve ?
		%
	% What is the significance of the problem issue or question ?
		%
%% Goal statement
	% What new understanding, knowledge, methods or technologies will this
	% research generate ?
		%
	% How this address the purpose of the work ?
		%
%% Approach
	% What experiments, prototypes or studies will be done to achieve the stated % goal ?
		%
	% How will achievement or contribution of the research be demonstrated or validated ?
		%

\chapterheader{Introduction}{Introduction}
\label{chap:introduction}

% Introducci�n al cap�tulo

\chaptertoc

\section{Introducci�n}
\label{sec:intr:introduction}

TODO: Siguiendo el esquema que aparece arriba, escribir la introducci�n

\section{Motivaci�n and Contribuciones}
\label{sec:intr:motivation}

TODO: Esta secci�n es m�s para tesis doctorales que para proyectos fin de carrera. La dejamos de momento pero se podr�a eliminar

\section{Visi�n General del Proyecto}
\label{sec:intr:overview}

TODO: Esto est� bien dejarlo, pero tambi�n es suprimible

\section{Estructura del Documento}
\label{sec:intr:organization}

Esto es una especie de �ndice ampliado y se deja, suele ser bastante �til para que el que est� vago se lea esto y se acabe el problema.

\paragraph{Cap�tulo 2: Resumen del Estado del Arte} \ \\

\paragraph{Cap�tulo 3: Descripci�n General del Proceso} \ \\

\paragraph{Cap�tulo 4: Definici�n y Planificaci�n del Proyecto} \ \\

\paragraph{Cap�tulo 5: Ingenier�a de Requisitos} \ \\

\paragraph{Cap�tulo 6: Definici�n Arquitect�nica y Dise�o Software} \ \\

\paragraph{Cap�tulo 7: Construcci�n e Implementaci�n} \ \\

\paragraph{Cap�tulo 8: Pruebas} \ \\

\paragraph{Cap�tulo 9: Despliegue y Aceptaci�n} \ \\

\paragraph{Cap�tulo 8: Discusi�n, Conclusiones y Trabajos Futuros} \ \\








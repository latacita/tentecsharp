\documentclass{sig-alt-full}

\usepackage{graphicx}

\begin{document}

\newcommand{\imp}[1]{{\small{\sf #1}}}
\newcommand{\stereotype}[1]{$<<${\small{\sf #1}}$>>$}

% --- Author Metadata here ---
\conferenceinfo{FOSD'10,}{October 13, 2010, Eindhoven (The Netherlands)}
\CopyrightYear{2010} \crdata{...}
% --- End of Author Metadata ---

\title{C\# Partial Classes as a Mechanisms to Implement Feature-Oriented Designs: An Exploratory Study\thanks{This work has been supported by the Spanish Ministry Project TIN2008-01942/TIN, the EC STREP Project AMPLE IST-033710, and the Junta de Andaluc{\'i}a regional project FarmWare TIC-5231.}}

\numberofauthors{3}

\author{
% 1st. author
\alignauthor Lidia Fuentes \\
       \affaddr{Dpto. Lenguajes y Ciencias de la Computaci{\'o}n}\\
       \affaddr{Universidad de M{\'a}laga (Spain)}\\
       \email{lff@lcc.uma.es}
% 2nd. author
\alignauthor Elio L�pez \\
       \affaddr{Dpto. Lenguajes y Ciencias de la Computaci{\'o}n}\\
       \affaddr{Universidad de M{\'a}laga (Spain)}\\
       \email{ealopezs@gmail.com}
% 3rd. author
\alignauthor Pablo S{\'a}nchez \\
       \affaddr{Dpto. Matem{\'a}ticas, Estad{\'i}stica y Computaci{\'o}n}\\
       \affaddr{Universidad de Cantabria (Spain)}\\
       \email{p.sanchez@unican.es}
}

\maketitle

\begin{abstract}
    %=========================================================================%
% Author: Pablo S�nchez                                                   %
% Paper: FOSD2010 (abstract)                                              %
% Version: 1.0                                                            %
% Date   : 2010/05/07                                                     %
%=========================================================================%

C\# partial classes allows developers to divide the implementation of a class into several slices where each slice contains an increment of functionality as compared to the other slices. Thus, combining different set of slices, we can get classes with a variable range of functionality. With this description, C\# partial classes seems to be, as also pointed out by other authors, a suitable mechanism for implementing feature-oriented designs. This paper explores this idea, by systematically applying C\# to a feature-oriented decomposition based on an industrial case study and comparing the results we have previously obtained using the feature-oriented language CaesarJ. As main contributions, (1) we identify benefits and pitfalls of C\# partial classes for implementing feature-oriented decompositions; and (2) we outline potential solutions to alleviate these pitfalls. 
\end{abstract}

% TODO: Cambiar esto
\category{D2.2}{Design Tools and Techniques}{}

% TODO: Cambiar esto
\terms{Design, Languages}

\keywords{Partial Classes, Feature-Oriented Programming, Software-Product Line}

%\begin{figure*}[!tb]
%  \begin{center}
%    \includegraphics[width=.80\linewidth]{images/robotExample.eps}
%    \caption{Robot example}
%    \label{fig:example}
%  \end{center}
%\end{figure*}


\section{Introduction}
\label{sec:introduction}

Pues eso, introduccion.

\section{What we want to find in a feature-oriented programming language}
\label{sec:fopFeatures}

B�sicamente, caracter�sticas que tienen los lenguajes como CaesarJ que nos facilitan la vida para implementar modelos de SPL, tales como lo de reescritura autom�tica de las dependencias entre clases.

\section{Case Study: A Smart Home Software Product Line}
\label{sec:caseStudy}

Copia y pega de la descripci�n de la Smart Home m�s breve explicaci�n de los modelos arquitect�nicos. La figura del modelo hay que ponerla despu�s de escribir el art�culo, para que contenga todos los elementos que nos hacen falta para explicar los fallos de C\#.

\section{Implementing the Smart Home models using C\# partial classes}
\label{sec:partialClasses}

Describir como se implementan diferentes situaciones en el modelo de la SmartHome.

\subsection{Scenarios 1: ...}

Descripci�n de como implementar algo tal como una feature que extiende de otra.

\subsection{Scenarios 2: ...}

Descripci�n de como implementar otra cosa, tal como una feature que extiende de dos.

\subsection{Scenarios 3: ...}

Otro caso diferente ...

\subsection{Conclusions}

Tabla o bullets comparando lo que querr�amos tener y lo que tenemos con las clases parciales.

\section{Sketch of solutions for C\# partial classes pitfalls}

Breves comentarios sobre como solucionar estos problemas. (NOTA: No estoy muy seguro acerca de si poner o no esta secci�n).

\section{Summary and Future Work}

\bibliographystyle{plain}
\bibliography{mise08}

\balancecolumns

%\section{Demo screenplay}
%
%\input{screenplay}
%
%% This section describes the different steps that we will carry out
%% during the demo presentation as well as the different features of
%% the tool that will be shown and explained. The demonstration is
%% divided in 3 three main blocks: (1) introduction and background; (2)
%% description of \Populo\ and its features; and (3) explanation of how
%% \Populo\ can be extended for supporting the execution of profiled
%% UML models.
%
%
%
%\balancecolumns
%
%\newpage
%
%\begin{figure*}[t]
%  \begin{center}
%    \includegraphics[width=.75\linewidth]{images/image00.eps}\\
%    \caption{The execution trace, the breakpoint and the object properties}
%  \end{center}
%\end{figure*}
%
%\begin{figure*}
%  \begin{center}
%    \includegraphics[width=.75\linewidth]{images/image01.eps}\\
%    \caption{Ready and blocked actions and the call stack}
%  \end{center}
%\end{figure*}
%
%\begin{figure*}
%  \begin{center}
%    \includegraphics[width=.75\linewidth]{images/image02.eps}\\
%    \caption{Instance view}
%  \end{center}
%\end{figure*}
%
%\begin{figure*}
%  \begin{center}
%    \includegraphics[width=.75\linewidth]{images/image03.eps}\\
%    \caption{The model view}
%  \end{center}
%\end{figure*}
%
%\begin{figure*}
%  \begin{center}
%    \includegraphics[width=.75\linewidth]{images/image04.eps}\\
%    \caption{Parallel execution flows}
%  \end{center}
%\end{figure*}

\end{document}

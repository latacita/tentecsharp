%==================================================================%
% Author : P�rez Ruiz, Alejandro                                   %
% Version: 1.0, 16/03/2011                                         %                                                                                    %                                                                  %
% Memoria del Proyecto Fin de Carrera                              %
% Cap�tulo Conclusiones y Trabajos Futuros                         %
%==================================================================%

\chapterheader{Conclusiones y Trabajos Futuros}{Conclusiones y Trabajos Futuros}

\label{chap:conclusiones}
%Introducci�n al cap�tulo

\chaptertoc
\section{Conclusiones}
%%% Qu� hace exactamente mi proyecto
Este proyecto ha desarrollado una l�nea de productos software para hogares autom�ticos y/o inteligentes mediante la tecnolog�a .NET en el lenguaje C\# a trav�s de sus clases parciales. El principal objetivo que se busca cuando se realiza un tipo de proyecto como este, es el de reducir costes de desarrollo y mejorar la calidad de los productos software. No obstante, el presente proyecto no ten�a como principal cometido desarrollar un l�nea de productos software que pudiese ser trasladada a un caso pr�ctico real de inmediato. En cierto modo, se ha utilizado un caso de estudio como es el de la automatizaci�n de los hogares, debido a la gran variabilidad y elementos comunes que poseen, para encontrar las posibles ventajas y desventajas que se presentan cuando se trabaja con las clases parciales de C\# sobre la tecnolog�a .NET para construir l�neas de productos software.

Durante el desarrollo del proyecto se han encontrado problemas que se presentan cuando se utilizan las clases parciales como mecanismo principal para encapsular las distintas caracter�sticas de una l�nea de productos. Por ello se han desarrollado nuevos mecanismos que pueden ser utilizados como una alternativa para suplir estos problemas. Por lo que el principal objetivo del proyecto ha sido demostrar que utilizando los mecanismos desarrollados en el presente proyecto se pueden construir l�neas de productos software utilizando .NET y las clases parciales de C\#.




%%% Qu� no hace

%%% Experiencias Personales

\section{Trabajos Futuros}

%%% Cualquier cosa que se puede hacer para mejorar este proyecto
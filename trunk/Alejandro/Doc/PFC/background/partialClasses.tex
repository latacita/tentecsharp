%=========================================================================%
% Author : Alejandro P�rez Ruiz                                           %
% Author : Pablo S�nchez Barreiro                                         %
% Version: 1.1, 10/06/2011                                                %
% Master Thesis: Background/PartialClasses                                %
%=========================================================================%

Las clases parciales C\# \cite{albahari:2010} permiten dividir la implementaci�n de una clase en varios archivos de c�digo fuente. Cada fragmento representa una parte de la funcionalidad global de la clase. Todos estos fragmentos se combinan en tiempo de compilaci�n para crear una �nica clase, la cual contiene toda la funcionalidad especificada en las clases parciales. Por lo tanto, las clases parciales C\# parecen un mecanismo adecuado para implementar caracter�sticas, tal como ha sido identificado por diversos autores~\cite{laguna:2007,laguna:2010}, dado que cada incremento en funcionalidad perteneciente a una caracter�stica se podr�a encapsular en una clase parcial separada.

%%==============================================================================================================%%
%% NOTA(Pablo): Poner un ejemplo peque�o aqu� con dos clases parciales relativas al problema de las expresiones %%
%%==============================================================================================================%%

%%=============================================================================================================%%
%% NOTA(Pablo): Esto no me cuadra muy bien aqu�, as� que lo borro                                              %%
%%=============================================================================================================%%
%%
%% Para poder ser compiladas y agrupadas en una sola clase, todas las clases parciales deben pertenecer al
%% mismo espacio de nombres, poseer la misma visibilidad y deben ser declaradas con la palabra clave
%% \imp{partial}. En C\#, un espacio de nombre se emplea simplemente para agrupar clases relacionadas y evitar
%% conflictos de nombres.
%%
%%=============================================================================================================%%


Para especificar los archivos C\# que deben ser incluidos en una unidad de compilaci�n se emplea un documento XML como el de la Figura~\ref{back:fig:partialClass}. Por lo tanto, es posible incluir y excluir f�cilmente la funcionalidad encapsulada dentro de una clase parcial simplemente a�adiendo o eliminando dicha clase parcial de este fichero XML. Por tanto, la composici�n de caracter�sticas tambi�n parece factible mediante la manipulaci�n de este fichero XML.

\begin{figure}[!h]
\begin{center}
\begin{footnotesize}
\begin{verbatim}
01    <itemgroup>
02    <!--Eval-->
03    <Compile Include="Eval\Add.cs" />
04    <Compile Include="Eval\IExpressionsEval.cs" />
05    <Compile Include="Eval\IExpressions.cs" />
06    <Compile Include="Eval\Integer.cs" />
07    <Compile Include="Eval\Mult.cs" />
08    <!--Infix-->
09    <!--<Compile Include="Infix\Add.cs" />
10    <Compile Include="Infix\IExpressionInfix.cs" />
11    <Compile Include="Infix\IExpressions.cs" />
12    <Compile Include="Infix\Integer.cs" />
13    <Compile Include="Infix\Mult.cs" />-->
14    ...
15    </itemgroup>
16    </Project>
\end{verbatim}
\end{footnotesize}
\end{center}
\caption{Archivo XML que guarda la informaci�n para la compilaci�n en Visual Studio}
\label{back:fig:partialClass}
\end{figure}

%%=============================================================================================================%%
%% NOTA(Pablo): Esto no me cuadra muy bien aqu�, as� que lo borro                                              %%
%%=============================================================================================================%%
%% Para ilustrar lo dicho anteriormente, se ha vuelto a utilizar el problema de las expresiones implement�ndolo
%% con clases parciales. La figura \ref{back:fig:partialClass} muestra como hemos excluido de la compilaci�n la
%% caracter�stica que representa la operaci�n de imprimir una expresi�n en formato infijo.
%% Este mecanismo de clases parciales permite a�adir o compartir funcionalidad entre un conjunto de clases que
%% no precisan estar relacionadas mediante ning�n tipo de relaci�n jer�rquica, tal como ocurre con la herencia.


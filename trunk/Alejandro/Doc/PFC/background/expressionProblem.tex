%=============================================================================%
% Author : Alejandro P�rez Ruiz                                               %
% Author : Pablo S�nchez Barreiro                                             %
% Version: 1.1, 10/06/2011                                                    %
% Master Thesis: Background/Expression Problem                                %
%=============================================================================%

El \emph{problema de las expresiones} es un problema t�pico dentro del mundo del dise�o software~\cite{cook:1990} y que es muy frecuentemente utilizado para ilustrar el funcionamiento de las diferentes t�cnicas y tecnolog�as relacionadas con las l�neas de productos software~\cite{lopezHerrejon:2004}. El objetivo del problema de las expresiones es dise�ar una familia de productos software que, para la gram�tica de la Figura~\ref{back:fig:gramExpr}, soporte siguientes operaciones:

\begin{description}
	\item[Print:] Debe mostrar por consola la expresi�n en el formato infijo, prefijo o posfijo.
	\item[Eval:] Debe evaluar la expresi�n y retornar su resultado.
	\item[ShortEval:] debe evaluar la expresi�n realizando las operaciones \emph{cortocircuitadas}. Es decir, tan pronto como el valor de un operando determine el resultado de la expresi�n, se deber� parar la evaluaci�n del resto de los operandos. Por ejemplo, en una multiplicaci�n, si el primer operando es 0, se retornar� el valor 0 directamente, sin evaluar el segundo operando.
\end{description}

No todas las operaciones tienen que aparecer en todos los productos, por lo que deber�a ser posible construir productos concretos que careciesen de alguna de ellas. 

\begin{figure}
\begin{center}
\begin{footnotesize}
\begin{verbatim}
Exp :: = Integer | AddInfix | MultInfix | AddPostfix | MulltPostfix |
				 AddPrefix | MultPrefix
Integer     :: <positive-negative integers>
AddInfix    ::= Exp "+" Exp
MultInfix   ::= Exp "*" Exp
AddPostfix  ::= Exp Exp "+"
MultPostfix ::= Exp Exp "*"
AddPrefix   ::= "+" Exp Exp
MultPrefix  ::= "*" Exp Exp
\end{verbatim}
\end{footnotesize}
\end{center}
\caption{Gram�tica del lenguaje de expresiones}
\label{back:fig:gramExpr}
\end{figure}

La siguiente secci�n describe c�mo usando �rboles de caracter�sticas podemos especificar la variabilidad existente en esta familia de productos software.


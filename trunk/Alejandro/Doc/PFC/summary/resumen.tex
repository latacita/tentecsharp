%=============================================================================%
% Author : Alejandro P�rez Ruiz                                               %
% Author : Pablo S�nchez Barreiro                                             %
% Version: 1.0, 30/05/2011                                                    %
% Version: 1.1, 06/05/2011                                                    %
% Master Thesis: Resumen                                                      %
%=============================================================================%



\cdpchapter{
\vspace{-80pt}Resumen}

\vspace{-20pt}
El objetivo de una \emph{l�nea de productos software} es crear una infraestructura a partir de la cual se puedan derivar, tan autom�ticamente como sea posible, productos concretos pertenecientes a una familia de productos software. Una \emph{familia de productos software} es un conjunto de aplicaciones software similares, que por tanto comparten una serie de caracter�sticas comunes, pero que tambi�n presentan variaciones entre ellos.

El software para el control de hogares automatizados es un claro ejemplo de dominio donde un enfoque de l�neas de productos software resulta muy adecuado. Dicho dominio presenta un amplio rango de variaciones debido a los diferentes dispositivos que pueden ser controlados (por ejemplo, ventanas, puertas, luces, etc.) y las funciones que se desea que dichos dispositivos realicen (por ejemplo, simulaci�n de presencia, control inteligente de la energ�a, etc.).

El objetivo de este proyecto es crear una l�nea de productos software para el desarrollo de software para hogares inteligentes de forma que se pueda automatizar el proceso de desarrollo de aplicaciones concretas.  Dicha l�nea de productos software ha sido desarrollada en la plataforma .NET, usando las clases parciales de C\# como principal mecanismo para soportar las diferentes variaciones existentes. La especificaci�n de requisitos que debe cumplir la l�nea de productos est� basada en un caso de estudio industrial proporcionado por Siemens AG.

Las ventajas de alas l�neas de productos software es que se pueden crear productos software concretos, pertenecientes a la familia de productos, de forma autom�tica mediante la simple especificaci�n de las caracter�sticas que se desea incluir en un producto concreto. Se reducen los tiempos de desarrollo y en consecuencia el coste asociado a cada producto concreto.

Como resultado del proyecto, se han desarrollado una serie de extensiones para  Visual Studio 2010. Dichos extensiones permiten modelar hogares autom�ticos y/o inteligentes seleccionando las caracter�sticas que mejor se adapten las necesidades del cliente. A continuaci�n, mediante generaci�n de c�digo, se obtiene autom�ticamente el producto deseado.

\paragraph{Palabras Clave} \ \\

L�nea de Productos Software, Desarrollo Software Orientado a Caracter�sticas, TENTE, Clases Parciales C\#, .NET


%=============================================================================%
% Author : Alejandro P�rez Ruiz                                               %
% Author : Pablo S�nchez Barreiro                                             %
% Version: 1.0, 30/05/2011                                                    %
% Version: 1.1, 06/05/2011                                                    %
% Master Thesis: Resumen                                                      %
%=============================================================================%

\cdpchapter{Resumen}

El objetivo de una \emph{l�nea de productos software}~\cite{pohl:2005} es crear una infraestructura adecuada a partir de la cual se puedan derivar, tan autom�ticamente como sea posible, productos concretos pertenecientes a una familia de productos software. Una \emph{familia de productos software} es un conjunto de aplicaciones software similares, que por tanto comparten una serie de caracter�sticas comunes, pero que tambi�n presentan variaciones entre ellos.

El software para el control de hogares inteligentes o automatizados es un claro ejemplo de dominio donde un enfoque de l�neas de productos software resulta muy adecuado. Dicho software presenta un amplio rango de variaciones debido a los diferentes dispositivos que pueden ser controlados en cada tipo de hogar (por ejemplo, ventanas, puertas, luces, radiadores, etc.) y las funciones que se desea que dichos dispositivos realicen (por ejemplo, simulaci�n de presencia, encendido y apagado autom�tico de luces, control inteligente de la energ�a, etc.).

El objetivo del proyecto es crear una l�nea de productos software para hogares autom�ticos o inteligentes de forma que se pueda automatizar el proceso de desarrollo de aplicaciones concretas para hogares automatizados.  Dicha infraestructura ha sido desarrollada en la plataforma .NET, usando las clases parciales de C\# como principal mecanismo para la modularizaci�n, composici�n y gesti�n de las diferentes caracter�sticas variables que conformar�n la l�nea de productos software. La especificaci�n de requisitos que debe cumplir la l�nea de productos est� basada en un caso de estudio industrial proporcionado por Siemens AG.

Las ventajas de �ste enfoque es que se pueden crear productos software concretos, pertenecientes a la familia de productos, de forma autom�tica mediante la simple  especificaci�n de las caracter�sticas que se desea incluir en cada producto concreto. Por tanto, se reducen dr�sticamente los tiempos de desarrollo y en consecuencia el coste asociado a la construcci�n de cada producto concreto.

Como resultado del proyecto, se han desarrollado una serie de extensiones para el entorno de desarrollo Visual Studio 2010. Dicho plugins permiten a los usuarios modelar hogares autom�ticos y/o inteligentes seleccionando las caracter�sticas que mejor se adapten a sus necesidades. A continuaci�n, mediante t�cnicas de generaci�n de c�digo, se obtiene autom�ticamente todo el c�digo correspondiente al producto software deseado.

\paragraph{Palabras Clave} \ \\

L�nea de Productos Software, Desarrollo Software Orientado a Caracter�sticas, TENTE, Clases Parciales C\#, .NET


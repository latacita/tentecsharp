%%==================================================================%%
%% Author : Tejedo GonzÁlez, Daniel                                 %%
%%          Sánchez Barreiro, Pablo                                 %%
%% Version: 1.0, 04/02/2013                                         %%
%%                                                                  %%
%% Memoria del Proyecto Fin de Carrera                              %%
%% Antecedentes, Árboles de características                         %%
%%==================================================================%%
El objetivo de las líneas de productos software es crear la infraestructura para la rápida  producción de sistemas software para un segmento de mercado específico, donde estos sistemas software son similares, y aunque comparten un subconjunto de características comunes, también presentan variaciones entre ellos ~\ref{} ~\ref{} ~\ref{} ~\ref{}.

El principal logro en las líneas de productos software es, construir productos específicos lo más automáticamente posible a partir de un conjunto de elecciones y decisiones adoptadas sobre un modelo común, conocido como modelo de referencia, que representa la familia completa de productos que la línea de productos software cubre.

El desarrollo de líneas de producto software se compone de dos procesos de desarrollo software diferentes pero íntimamente relacionados, conocidos como ingeniería del dominio e ingeniería de la aplicación.

En el nivel de ingeniería del dominio, comenzamos por los documentos de requisitos que describen una familia de productos similares para un segmento de mercado específico. Entonces, diseñamos una arquitectura e implementación de referencia para esta familia de productos. Esta arquitectura de referencia contiene los elementos que son comunes para todos los productos de la familia.

En el nivel de ingeniería de la aplicación, comenzamos un documento de requisitos de un producto específico. Este documento estable las variaciones específicas que deben ser incluidas en este producto concreto. Con esta información, introducimos los cambios en la arquitectura y en la implementación de referencia, y se debería obtener como resultado un producto software único.

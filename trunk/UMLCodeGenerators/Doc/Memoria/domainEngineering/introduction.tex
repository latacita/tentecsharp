%%==================================================================%%
%% Author : Abascal Fernández, Patricia                             %%
%% Author : Sánchez Barreiro, Pablo                                 %%
%% Version: 1.4, 21/06/2013                                         %%
%%                                                                  %%
%% Memoria del Proyecto Fin de Carrera                              %%
%% Domain Engineering/Transformación UML a C#                       %%
%%==================================================================%%

\todo{Escribir una pequeña intro para el capítulo, que no es más que igual que la pequeña intro del documento, pero ampliada, donde se describe con mayor detalle la metodología seguida en esta parte del proyecto}

El primer paso a la hora de desarrollar un generador de código es establecer una serie de correspondencias entre los distintos tipos de elementos que pueden aparecer en los modelos que sirven como entrada y los elementos del lenguaje de programación destino. En nuestro caso, se trata de establecer una correspondencia entre elementos UML 2.0 y el lenguaje C\#, teniendo en cuenta que los elementos de entrada como los de salida deben seguir un enfoque orientado a características.



%%==================================================================%%
%% Author : Abascal Fern�ndez, Patricia                             %%
%%          S�nchez Barreiro, Pablo                                 %%
%% Version: 1.2, 17/06/2013                                         %%                                                                                    %%                                                                  %%
%% Memoria del Proyecto Fin de Carrera                              %%
%% Archivo ra�z                                                     %%
%%==================================================================%%

\cdpchapter{Preface}

Within the department of Mathematics, Statistics and Computing has been previously developed a number of techniques for implementing and configuring software product lines on the .NET platform based on C\# partial classes. Said techniques are condensed in the so-called \emph{Slicer Pattern}. Nevertheless, the application of this pattern manually involves a number of manual and repetitive tasks.

This project aims to develop a number of code generators which allow to automate the \emph{Slicer Pattern} application. This will decrease the development time; and, therefore, the cost. Moreover, by automating the process, it avoids the insertion of errors due to human intervention. This contributes to increase the quality of the final product and reduce development time and costs; because the time required to detect and correct these potential errors disappear. 

To achieve this goal, this project has developed a number of code generators which transform design models, in UML 2.0, of a software product line in a C\#-based implementation based on the \emph{Slicer Pattern}. These code generators has been implemented using EGL (\emph{Epsilon Generation Language}), model to text transformation language from \emph{Epsilon}'s model manipulation suite tool.

\paragraph{Keywords} \ \\

Software Product Line, Code Generation, Feature-Oriented Software Development, C\# Partial Classes, Slicer Pattern, .NET, Epsilon, Te.NET, TENTE. 

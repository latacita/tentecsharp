%=============================================================================%
% Author : Alejandro P�rez Ruiz                                               %
% Author : Pablo S�nchez Barreiro                                             %
% Version: 1.0, 30/05/2011                                                    %
% Version: 1.1, 06/06/2011                                                    %
% Master Thesis: Resumen                                                      %
%=============================================================================%

\cdpchapter{Preface}

The goal of a software product line is to create an adequate infrastructure from which specific products within a software products family
can be constructed as automatically as possible. A \emph{family of software products} is a set of similar software applications, which share some common characteristics, but they also present variations between them.

The software to control a smart home is an typical domain where a software product line approach is particularly suitable. This domain provides a wide range of variations due to the different devices to be controlled in each specific software setup (e.g., windows, doors, lights, heaters, etc.) and the functions these devices must execute (e.g., presence simulation, automatic light control, smart energy control, etc.).

This project aims to construct a software product line for automated houses. The final goal is to automate the development process of specific applications belonging to this family of products. This software product line has been developed on the platform .NET. C\# Partial classes have been used as the main mechanism for modularization, composition and management of the different variable features we have found inside this family of software products. The requirements specification for our software product line is based on an industrial case study provided by Siemens AG.

The main benefit of creating such a software product line is to be able to create specific software products automatically by means of simply specifying which features must be included in each software product. This reduces the software development effort, decreasing the cost of each specific product.

This project has produced as a result several extensions to Visual Studio 2010. These plugins allow developers to visually specify which features must be included in a specific product. Then, all the code required for running such a product is automatically obtained using automatic code generation.

\paragraph{Keywords} \ \\

Software Product Lines, Feature-Oriented Software Development, TENTE, C\# Partial Classes, .NET

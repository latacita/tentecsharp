%%==========================================================================%%
%% Author : Abascal Fern�ndez, Patricia                                     %%
%% Author : S�nchez Barreiro, Pablo                                         %%
%% Version: 1.1, 15/05/2013                                                 %%
%%                                                                          %%
%% Memoria del Proyecto Fin de Carrera                                      %%
%% Application Engineering/Pruebas                                          %%
%%==========================================================================%%

Una vez implementados los generadores de c�digo para la fase de \emph{Ingenier�a de Aplicaciones}, el siguiente paso era dise�ar y ejecutar las pruebas necesarias que permitiesen comprobar el correcto funcionamiento de estos generadores. Para dise�ar las pruebas,  siguiendo el mismo procedimiento que en el caso anterior, utilizando la t�cnica de clases de equivalencia y valores l�mites, para luego completar con casos espec�ficos que permitiesen alcanzar el 100\% de la cobertura.

A diferencia de la fase de \emph{Ingenier�a del Dominio}, en esta fase no se utiliz� \emph{EUnit} para ejecutar dichas pruebas, ya que dicha herramienta no se ajustaba a nuestras necesidades. Por tanto, se crearon los casos de prueba y se ejecutaron a mano, analizando de forma tambi�n manual si la salida producida coincid�a con la esperada. La Tabla~\ref{app:table:pruebas} muestra algunos de los casos de prueba ejecutados.

\begin{table}
\begin{small}
\begin{tabularx}{\linewidth}{|X|l|}
 \hline
{Casos v�lidos}&{Casos no v�lidos} \\ \hline
Configuraci�n con un solo camino hoja-raiz & Paquetes recursivos. \\
Configuraci�n con varios caminos y todos los m�todos independientes &\\
Configuraci�n con varios caminos y alg�n m�todo dependiente &\\
Configuraci�n con varios caminos, donde alg�n m�todo tiene versiones dependientes e independientes &\\
\hline
\end{tabularx}
\end{small}
\caption{Casos de prueba para los generadores de Ingenier�a de la Aplicaci�n}
\label{app:table:pruebas}
\end{table}%

Una vez creados y probados los generadores de c�digo, el siguiente paso era empaquetarlos para posibilitar su distribuci�n y uso. La siguiente secci�n describe como se realiza dicha fase de despliegue.
Tras ejecutar estos casos de prueba y comprobar que los generadores de c�digo funcionaban correctamente, d�bamos por concluida la labore de desarrollo de los generadores de c�digo, restando solo su empaquetado y despliegue.


%%==================================================================%%
%% Author : Abascal Fernández, Patricia                             %%
%% Author : Sánchez Barreiro, Pablo                                 %%
%% Version: 1.0, 24/06/2013                                         %%
%%                                                                  %%
%% Memoria del Proyecto Fin de Carrera                              %%
%% Application Engineering/Modelo de Entrada                        %%
%%==================================================================%%

La estrategia para crear un modelo arquitectónico concreto consiste, tal como se describió en la Sección~\ref{}, en crear un paquete vacío, el cual representa el producto a construir, y añadir relaciones \emph{merge} a aquellas características que se desean incluir en el producto final. De esta forma, el contenido de los paquetes correspondiente a características que se deben incluir en el producto final, se combinan o componen en el paquete que representa el producto final.

Para distinguir el paquete que representa el producto final de los paquetes que representan características, se ha creado un perfil de UML 2.0~\citep{omg:uml:2005}. Un perfil UML 2.0 es un mecanismo genérico de extensión que permite personalizar los modelos UML para un propósito particular, mediante la especificación de estereotipos y valores etiquetados que modifican la semántica original de los elementos del modelo UML 2.0.

\begin{figure}[!tb]
  \center
  \includegraphics[width=\linewidth]{applicationEngineering/images/configuracion(1).eps} \\
  \caption{Configuración de un hogar inteligente completo}
  \label{app:fig:conf1}
\end{figure}

En nuestro caso, el perfil contiene un solo estereotipo, denominado \imp{SpecificProduct}, el cual se puede aplicar exclusivamente a paquetes UML tal como se puede apreciar en la Figura\ref{app:fig:conf1}. Además, por cada modelo UML 2.0 representando un producto concreto, sólo puede existir un paquete estereotipado de dicha forma. Esta última restricción se expresa por medio de OCL.

La Figura~\ref{app:fig:conf1} muestra un ejemplo de creación de un producto concreto dentro de la línea de productos software para hogares inteligentes. En este caso, se trata de un producto donde se ha incluido exclusivamente la característica de \imp{SmartEnergyMng}, lo que implica que deben seleccionarse además las características \imp{WindowMng} y \imp{HeaterMng}, ya que \imp{SmartEnergyMng} necesita que ambas características estén instaladas en un producto final para poder funcionar. Dicha dependencia queda especificada de forma explícita a través de las relaciones \emph{merge} existentes entre \imp{SmartEnergyMng} y \imp{WindowMng} y \imp{HeaterMng}. Debido a dichas relaciones, es imposible crear un producto que incluya \imp{SmartEnergyMng} pero no \imp{WindowMng} o \imp{HeaterMng}.

La siguiente sección describe como este modelo arquitectónico puede transformarse automáticamente en el código necesario para crear una implementación concreta y completamente funcional de un producto software concreto.


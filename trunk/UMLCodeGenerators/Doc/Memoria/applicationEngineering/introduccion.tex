%%==================================================================%%
%% Author : Abascal Fernández, Patricia                             %%
%% Author : Sánchez Barreiro, Pablo                                 %%
%% Version: 1.2, 24/06/2013                                         %%
%%                                                                  %%
%% Memoria del Proyecto Fin de Carrera                              %%
%% Application Engineering/Introduccion                             %%
%%==================================================================%%

\todo{Escribir una intro similar a la del capítulo anterior. Te dejo algunas frases}

El primer paso para crear un producto concreto, de acuerdo con la metodología Te.Net (ver Sección~\ref{}) es crear una selección de aquellas características que se desea incluir en el producto. Cómo se crea dicha selección de características está fuera del ámbito de este proyecto fin de carrera. Referimos al lector interesado en tal asunto a otros proyectos fin de carrera presentados en esta misma Facultad sobre dicho tema~\citep{}.

Una vez que se tiene una selección de características válida, utilizando dicha selección de características, se configura la arquitectura de referencia creada en la fase de Ingeniería del Dominio para crear un modelo arquitectónico concreto, adaptado a las necesidades del cliente, del producto que queremos construir.  Dicho modelo arquitectónico se obtiene de forma automática mediante la utilización del lenguaje \emph{VML}~\citep{}, de acuerdo con la metodología Te.Net (ver Sección~\ref{}).




%=========================================================================%
% Author: Pablo S�nchez                                                   %
% Paper: FOSD2010 (FOP Features)                                          %
% Version: 1.0                                                            %
% Date   : 2010/05/07                                                     %
%=========================================================================%

This section describes which characteristics are desirable to find in a language to
support feature-oriented programming.

% Try to find better supporting references
Feature-Oriented Programming~\cite{prehofer:1997} aims to encapsulates coherent slices of the functionality provided by an application into independent and composable modules called \emph{features}. Therefore, different versions of a same application should be easily obtained by simply combining different set of features. Obviously, not all feature combinations lead to right final applications, so feature-oriented languages should try to ensure the result of composing a set of features produces a safe and correct application. For instance, for the previous

Wit these goals on mind, and based on~\cite{Herrejon:2005,kang:2002,chae:2009,aracic:2006}, we have identified several desirable characteristics we would like to find in a language for supporting feature-oriented programming. We comment on each one of them.

\paragraph{Feature Encapsulation and Extensibility} \ \\

A feature is often considered as an increment on program functionality~\cite{zave:1999,Herrejon:2005}. For instance, in the SmartHome case study, \imp{LightMng}, \imp{WindowMng}, and so forth, are increments on functionality for monitoring and controlling new devices. Thus, feature-oriented languages must provide mechanism for adding new functionality to the existing one. In object-oriented languages, extension is achieved by inheritance. Inheritance is adequate when we need to extend a single class, but commonly, features need to extend several classes at the same time. For instance, the \imp{LightMng} feature needs to add a new class for the light controller (\imp{LightCtrl}) and it needs to add to the \imp{Gateway} class methods for switching on and off lights. So, a language with feature-oriented support must provide extension mechanisms.

Extension is not always achieved by addition, sometimes \emph{substitution} is required. For instance, in the case of the \imp{SmartEnergyMng} feature, it is necessary, for instance, to override the implementation of the method \imp{adjustTemperature} for checking is additional operations are required. For instance, if the command is to cool a room with the windows closed and the outside temperature is lower than the indoor one, windows will be open in addition to switching off heaters. This kind of substitution is achieved in object-oriented languages by method overriding.

All the extensions belonging to a certain feature must be added in a atomic way, i.e. either all extensions are added or no extension is carried out at all. For instance, the \imp{Gateway} class must not be extended with the
\imp{swicthLight} method if the class \imp{Light} is not added at all. Therefore, featured-oriented languages should provide mechanisms to group and ideally encapsulate elements belonging to a same feature in well-identified modules. Moreover, these modules should be compilable separately. In Figure~\ref{fig:SH-DM}, we use UML packages with this purpose.

Finally, extensions might have unavoidable ripple effects. For instance, sensors might have a reference to the \imp{Gateway} class to send messages with urgent or critic situations are detected, such as fire, for instance. The class \imp{Gateway} is refined through each feature, thus, several versions of the \imp{Gateway} class are created, e.g. \imp{LightMng::Gateway}, \imp{HeaterMng::Gateway}, \imp{HeaterMng::Gateway} and so forth. Thus, depending upon the selected features the customer want to include in the final product, a specific subclass, or combination of subclasses, should be selected. Let us suppose we select the \imp{LightMng} feature. In that case, the version of the \imp{Gateway} to be included in the final product is \imp{LightMng::Gateway}. But, the \imp{Sensor} class references \imp{InitialModel::Gateway}. So, we should update constructors, getters, setters and so forth to reference the subclass, i.e. \imp{LightMng::Gateway}, and get rid of castings and other problems related to the type system. Usually, this need to be done manually. Nevertheless, some languages, such as CeasarJ~\cite{aracic:2006} or ObjectTeams~\cite{Herrmann:2007} are able to automatically update these dependencies due to the usage of an advanced type system based on virtual clases and mixin composition. So, automatic dependency management is also a desirable characteristic in feature-oriented languages.

\paragraph{Feature-Level Composition and Composition Consistency Checking} \ \\

Specific products, or configurations of a feature-oriented decomposition, are obtained by means of selecting a set of features. Thus, it would desirable a feature-oriented language would provide language constructs for specifying which specific features should be instantiated, or composed, when producing a specific product. This should be done at the feature-level, specifying which \emph{feature modules} should be included, instead of having to specify individual elements of each feature. For instance, if the \imp{LightMng} feature has been selected,
we would like to specify the \imp{LightMng} package should be included, according to Figure~\ref{fig:SH-DM}, instead of having to specify the \imp{LightMng::Gateway} feature and the \imp{LightCtrl} class must be added individually to the final product.

Moreover, neither all feature combinations can be composed safely nor all composition ordering are valid. For instance, if the \imp{SmartEnergyMng} feature is selected, the \imp{WindowMng} and \imp{HeaterMng} features should also be selected. Moreover, \imp{WindowMng} and \imp{HeaterMng} must be composed before \imp{SmartEnergyMng} is composed, since the latter one depends on the former ones. A feature-oriented language should be aware of these constraints, avoiding invalid product can be created and managing automatically dependencies whenever possible. For instance, when \imp{SmartEnergyMng} feature is selected, the language should detect the \imp{WindowMng} and \imp{HeaterMng} should be also composed and, in addition, before the \imp{SmartEnergyMng} feature.

So summarising, a feature-oriented language must provide:

\begin{enumerate}
    \item Feature Extension by addition and substitution.
    \item Modules to group and encapsulate feature elements.
    \item Automatic dependency management at intra-feature level.
    \item Composition at the feature level.
    \item Composition consistency checking.
    \item Automatic management of constraints at the feature level.
\end{enumerate}

Next section will evaluate how C\# partial classes are able to manage these issues. 
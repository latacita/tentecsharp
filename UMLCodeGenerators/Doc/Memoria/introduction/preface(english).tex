%%==================================================================%%
%% Author : Abascal Fern�ndez, Patricia                             %%
%%          S�nchez Barreiro, Pablo                                 %%
%% Version: 1.3, 18/06/2013                                         %%                                                                                    %%                                                                  %%
%% Memoria del Proyecto Fin de Carrera                              %%
%% Archivo ra�z                                                     %%
%%==================================================================%%

\cdpchapter{Preface}

During the last years, a family of languages, tools and techniques for the development and configuration of software product lines for the .NET platform has been released in the Departamento de Matem�ticas, Estad�stica y Computaci�n of the Universidad de Cantabria. One of the most emergent techniques is what is known as the \emph{Slicer Pattern}, which allows developers to achieve some degree of support for feature-oriented programming in C\#. However, the manual instantiation of this pattern has associated a considerable effort, as it implies a large number of repetitive, laborious and error-prone tasks. 

This capstone projects aims to overcome this limitation by means of the development of a set of code generators that automate the instantiation of the \emph{Slicer Pattern}. This will reduce development efforts, and, therefore, development time and cost. Moreover, this automation avoids errors due to human intervention, which contributes to increase the product quality and to reduce development time and costs even more; since the time required to detect and correct these errors is saved.

Thus, this project has developed several code generators which transform design models, in UML 2.0, of a software product line into a C\#-based implementation, based on the \emph{Slicer Pattern}, of such software product line. These code generators have been implemented using EGL (\emph{Epsilon Generation Language}), the model to text transformation language of \emph{Epsilon} suite  for model driven software development.

\paragraph{Keywords} \ \\

Software Product Line, Code Generation, Feature-Oriented Software Development, C\# Partial Classes, Slicer Pattern, .NET, Epsilon, Te.NET, TENTE.

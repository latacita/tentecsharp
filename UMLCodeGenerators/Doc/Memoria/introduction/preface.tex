%%==================================================================%%
%% Author : Abascal Fernández, Patricia                             %%
%%          Sánchez Barreiro, Pablo                                 %%
%% Version: 1.3, 18/06/2013                                         %%                                                                                    %%                                                                  %%
%% Memoria del Proyecto Fin de Carrera                              %%
%% Archivo raíz                                                     %%
%%==================================================================%%

\cdpchapter{Resumen}
Dentro del Departamento de Matemáticas, Estadística y Computación se han desarrollado con anterioridad una serie de técnicas para la implementación y configuración de líneas de productos software para la plataforma .NET basándose en las clases parciales de C\#. Dichas técnicas se condensan en el denominado \emph{Slicer Pattern}. No obstante, la aplicación de dicho patrón de forma manual implica una serie de tareas manuales y repetitivas.

El objetivo de presente proyecto fin de carrera es desarrollar una serie de generadores de código que permitan automatizar la aplicación del \emph{Slicer Pattern}. Ello reduciría los tiempos de desarrollo; y , por tanto, el coste. Además, al automatizarse el proceso se evita la introducción de errores debidos a la intervención humana. Esto contribuye a aumentar la calidad del producto final y a reducir los tiempos y costes de desarrollo; ya que el tiempo necesario para detectar y corregir estos potenciales errores desaparece.

Para alcanzar dicho objetivo, este proyecto fin de carrera ha desarrollado una serie de generadores de código que transforman modelos de diseño, en UML 2.0, de una línea de productos software en una implementación en C\# basada en el \emph{Slicer Pattern}. Dichos generadores de código se han implementado usando EGL (\emph{Epsilon Generation Language}), el lenguaje de transformación modelo a código de la suite de herramientas para la manipulación de modelos \emph{Epsilon}.

\paragraph{Palabras Clave} \ \\

Línea de Productos Software, Generación de Código, Desarrollo Software Orientado a Características, Clases Parciales C\#, Patrón Slicer, .NET, Epsilon, Te.NET, TENTE.




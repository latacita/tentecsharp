%%==================================================================%%
%% Author : Tejedo Gonz�lez, Daniel                                 %%
%%          S�nchez Barreiro, Pablo                                 %%
%% Version: 1.0, 28/11/2012                                         %%                   %%                                                                  %%
%% Memoria del Proyecto Fin de Carrera                              %%
%% Validation Framework, archivo ra�z                                       %%
%%==================================================================%%

\chapterheader{Validaci�n de las sintaxis concretas}{Validaci�n de las sintaxis concretas}
\label{chap:emfvf}

Este cap�tulo trata de describir el desarrollo de la tarea de implementaci�n del proceso de validaci�n de las sintaxis concretas. La validaci�n consiste en tratar de averiguar si ciertos aspectos de la sintaxis concreta construida que no se pueden comprobar mediante la gram�tica y el metamodelo son correctos. Un ejemplo de uno de esos aspectos es si la direcci�n introducida en el import para el modelo de caracter�sticas es correcta.

\chaptertoc

\section{Captura de requisitos}
\label{sec:emfvf:req}
%%==================================================================%%
%% Author : Tejedo Gonz�lez, Daniel                                 %%
%%          S�nchez Barreiro, Pablo                                 %%
%% Version: 1.0, 27/11/2012                                         %%                   
%%                                                                  %%
%% Memoria del Proyecto Fin de Carrera                              %%
%% Gram�tica,  requisitos                                           %%
%%==================================================================%%

La captura de requisitos de la gram�tica pasa por informarse sobre qu� tipo de sintaxis textual queremos que tengan nuestras operaciones, lo cual en este caso ya estaba especificado previamente por documentos creados en iteraciones previas del proyecto Hydra. Lo m�s l�gico e inmediato era mantenerse fiel a esa sintaxis, seg�n la cual las operaciones deber�an ser expresadas textualmente tal como sigue: \\

Suma: operando1 + operando2

Resta: operando1 - operando2

Multiplicaci�n: operando1 * operando2

Divisi�n: operando1 / operando2

And: operando1 and operando2

Or: operando1 or operando2

Xor: operando1 xor operando2

Implica: operando1 implies operando2

Mayor que: operando1 > operando2

Menor que: operando1 < operando2

Igual que: operando1 == operando2

Mayor o igual que: operando1 >= operando2

Menor o igual que: operando1 <= operando2

Distinto que: operando1 != operando2

Contexto: operando1 [ operando2 ]

Para todo: all operando1 [ operando2 ]

Existe: any operando1 [ operando2 ] \\

Otros aspectos concernientes a la gram�tica que ha habido que tener en cuenta dentro de la fase de captura de requisitos son los siguientes:\\

- Ha de permitirse la posibilidad de especificar prioridad en las operaciones, es decir, de poder delimitar las operaciones con par�ntesis que denoten el orden de realizaci�n de las mismas.

- Todas las representaciones textuales de nuestro lenguaje han de empezar con una l�nea de carga del modelo de caracter�sticas al que han de aplicarse las restricciones. La sintaxis de este aspecto ser� "import operando", donde operando es la direcci�n del fichero hydra del modelo dentro del disco duro del sistema.

- Todas las restricciones definidas han de separarse entre ellas mediante el car�cter '' ; ''.\\

Por supuesto, adem�s de los requisitos aqu� expuestos tambi�n habr� que tener en cuenta todo lo comentado en el apartado de requisitos del cap�tulo anterior, ya que tambi�n influir�n a la hora de tomar decisiones de dise�o en la gram�tica.


\section{Implementaci�n de la validaci�n}
\label{sec:emfvf:req}
%%==================================================================%%
%% Author : Tejedo Gonz�lez, Daniel                                 %%
%%          S�nchez Barreiro, Pablo                                 %%
%% Version: 1.0, 28/11/2012                                         %%
%% Version: 2.0, 06/02/2013                                         %%
%%                                                                  %%
%% Memoria del Proyecto Fin de Carrera                              %%
%% Validation Framework, implementacion                             %%
%%==================================================================%%

La sintaxis propia de Ecore no nos permite especificar ciertas restricciones que debe satisfacer nuestro metamodelo. Dichas restricciones, las cuales se enumeran a continuaci�n, deben comprobar que:

\begin{enumerate}
    \item La ruta que indica donde est� el �rbol de caracter�sticas al que se
        aplican las restricciones definidas es correctas. Ello implica comprobar tanto que la ruta es correcta como que el fichero que se halla en dicha ruta corresponde de verdad a un �rbol de caracter�sticas.
    \item El atributo \emph{featureName} asociado a una \emph{SimpleFeature} o a una \emph{MultipleFeature} corresponde al nombre de una caracter�stica perteneciente al �rbol de caracter�sticas referenciado.
    \item Una caracter�stica identificada como \emph{SimpleFeature} en el modelo de restricciones (eval�a a verdadero o falso) es realmente una caracter�stica simple (no clonable) en el �rbol de caracter�sticas asociado. Sin esta comprobaci�n, podr�amos, por ejemplo, introducir caracter�sticas m�ltiples como operandos de operadores booleanas como \emph{and} u \emph{or}. En ese caso, ser�a imposible evaluar dichas operaciones ya que no podemos evaluar sus operandos a verdadero o falso.
\end{enumerate}

Respecto a la segunda restricci�n, merece la pena aclarar que el caso contrario, comprobar que una caracter�sticas considerada como m�ltiple en una restricci�n realmente lo sea, no es necesario controlarlo. La raz�n es que las caracter�sticas simples pueden tratarse como un caso particular o subtipo de caracter�stica m�ltiples, ya que siempre podremos considerarla como una caracter�stica clonable con cardinalidad $<0,1>$.

Para implementar estas restricciones externas se ha utilizado \emph{EMF Validation Framework}, herramienta integrada en EMF para este prop�sito concreto. Siguiendo las instrucciones proporcionadas por esta herramienta, a�adimos un m�todo de validaci�n a cada metaclase que necesitaba ser validada. En nuestro caso, dichas clases eran \emph{Model}, \emph{SimpleFeature} y \emph{MultipleFeature}. De acuerdo con las normas establecida por \emph{EMF Validation Framework}, dichos m�todos deben poseer un perfil concreto. Dicho perfil est� compuesto por dos par�metros, uno llamado \emph{diagnostics} del tipo \emph{EDiagnosticChain} y otro llamado \emph{context} que es un mapa \todo{cual es la clave y cual es valor de este mapa?}. Los m�todos de validaci�n han de retornar siempre un valor booleano.

Si la validaci�n es satisfactoria, el m�todo debe obviamente devolver un valor verdadero. En caso contrario, retornar� falso. El par�metro \emph{diagnostics} es un par�metro de salida que almacena informaci�n sobre el resultado de la validaci�n, tal como el tipo de error producido o el mensaje de error que queremos mostrar al usuario.

%%==========================================================================================%%
%% NOTA(Pablo): Explicar para qu� sirve el mapa                                             %%
%%==========================================================================================%%

Para implementar la primera restricci�n de las comentadas anteriormente, validar que la ruta que indica el �rbol de caracter�sticas sea v�lida y apunte realmente a un �rbol de caracter�sticas, se a�adi� un m�todo de validaci�n a la clase \emph{Model}. Para llevar a cabo esta validaci�n simplemente se carga el fichero existente en la direcci�n indicada y se controla las posibles excepciones que una direcci�n err�nea pueda generar. Adem�s, se comprueba que el contenido de esa direcci�n sea un �rbol de caracter�sticas. Se aprovecha tambi�n para generar una variable global que contenga el modelo le�do, ya que ser� necesario volver a cargarlo en posteriores comprobaciones.

Para implementar la segunda restricci�n, comprobar que la existencia de las caracter�sticas escritas en nuestro fichero de restricciones en el �rbol de caracter�sticas anteriormente asociado, a�adimos m�todos de validaci�n a las metaclases \emph{MultipleFeature} y \emph{SimpleFeature}. Para ello simplemente buscamos que el nombre almacenado en el par�metro \emph{featureName} de dichas metaclases corresponda con el nombre de alguna caracter�sticas del modelo cargado anteriormente. 

Para implementar la segunda restricci�n, que las caracter�sticas identificadas como simples realmente sean realmente simples en el �rbol de caracter�sticas asociado, a�adimos un m�todo de validaci�n a la clase \emph{SimpleFeature}. Para realizar la comprobaci�n tenemos que corroborar que �sta no pueda ser instanciada en m�s de una ocasi�n. Para ello tenemos que calcular la cota superior de su cardinalidad. Si dicha cota fuese mayor que uno, no ser�a una caracter�sticas simple. Este l�mite puede ser superior a uno en el caso de las caracter�sticas clonables, o de la caracter�sticas hijas de caracter�sticas m�ltiples. 

Tras a�adir estas restricciones estaba definida la sintaxis abstracta para nuestro lenguaje. Antes de proceder a la definici�n de una sintaxis concreta para dicho lenguaje, realizamos una serie de pruebas destinadas a verificar que el metamodelo creado recoge la sintaxis abstracta deseada. 

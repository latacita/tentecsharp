%%==================================================================%%
%% Author : Abascal Fern�ndez, Patricia                             %%
%% Author : S�nchez Barreiro, Pablo                                 %%
%% Version: 1.5, 15/05/2013                                         %%
%%                                                                  %%
%% Memoria del Proyecto Fin de Carrera                              %%
%% Conclusiones, trabajos futuros                                   %%
%%==================================================================%%

El proyecto desarrollado, aunque pequemos de falta de modestia, se trata de un proyecto bastante completo que deja pocos flecos por completar. De hecho, las cuestiones que nos han quedado por resolver han sido m�s debido a problemas con herramientas desarrolladas por terceros y que est�n fuera de nuestro alcance, que a problemas que podamos resolver por nosotros mismos. 

En concreto, una cuesti�n obvia que se debe resolver en cuanto los desarrolladores de Epsilon solucionen los problemas planteados es
completar la integraci�n con Eclipse, produciendo un plug-in que pueda ejecutar a los generadores de c�digo sin que estos generen ning�n tipo de excepci�n.

Como tareas m�s avanzadas, se podr�a plantear la integraci�n de los generadores de c�digo en \emph{Visual Studio 2010}, a trav�s de un complejo sistema de \emph{wrappers} y servicios sobre componentes OSGi, o idear mecanismos que permitan la verificaci�n formal de los generadores de c�digo, de forma que se pueda demostrar matem�ticamente que el proceso de generaci�n de c�digo preserva o mantiene una serie de invariantes identificados a nivel arquitect�nico. 
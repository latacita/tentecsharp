%%==================================================================%%
%% Author : Abascal Fern�ndez, Patricia                             %%
%% Author : S�nchez Barreiro, Pablo                                 %%
%% Version: 1.2, 24/06/2013                                         %%
%%                                                                  %%
%% Memoria del Proyecto Fin de Carrera                              %%
%% Application Engineering/Introduccion                             %%
%%==================================================================%% 

El primer paso para crear un producto concreto, de acuerdo con la metodolog�a Te.Net (ver Secci�n~\ref{sec:intr:tenet}) es crear una selecci�n de aquellas caracter�sticas que se desea incluir en el producto. C�mo se crea dicha selecci�n de caracter�sticas est� fuera del �mbito de este proyecto fin de carrera. Referimos al lector interesado en tal asunto a otros proyectos fin de carrera presentados en esta misma Facultad sobre dicho tema~\citep{}.

Una vez que se tiene una selecci�n de caracter�sticas v�lida, utilizando dicha selecci�n de caracter�sticas, se configura la arquitectura de referencia creada en la fase de Ingenier�a del Dominio para crear un modelo arquitect�nico concreto, adaptado a las necesidades del cliente, del producto que queremos construir.  Dicho modelo arquitect�nico se obtiene de forma autom�tica mediante la utilizaci�n del lenguaje \emph{VML}~\citep{}, de acuerdo con la metodolog�a Te.Net (ver Secci�n~\ref{sec:intr:tenet}).



